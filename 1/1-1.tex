\documentclass{article}
\usepackage{ben}
\begin{document}
\begin{enumerate}
\item[1.1.20.1]
  An arbitrary set $\mathcal{S}$
  may be considered as a (discreet) category by the following construction:
  \begin{itemize}
    \item
      The objects are the elements of the set
    \item 
      The arrows are identity arrows, mapping each object to itself. 
  \end{itemize}
  Each arrow $id_x$ for $x\in\mathcal{S}$ has domain and codomain $x$.
  The identity criterion for each object is satisfied by construction,
  and arrow composition and its associativity follow trivially.

  Another possible construction is to add arrows from each object to
  every other to the above construction. The identity arrows are preserved,
  and we have that for any two arrows $f:x\rightarrow y$, $g:y\rightarrow z$
  between elements $x,y,z\in\mathcal{S}$, 
  composition of arrows yields an arrow from $x$ to $z$. 

%% \item[1.1.20.2]
%%   An arbitrary group $G$ can be considered as a category by 
%%   taking:
%%   \begin{itemize}
%%     \item
%%       Elements $g\in G$ as objects
%%     \item

      
\end{enumerate}
\end{document}
