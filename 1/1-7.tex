\documentclass{article}
\usepackage{ben}
\begin{document}
\begin{enumerate}
\item[1.7.2]
  An equalizer $e : X \rightarrow A$ for arrows $f: A \rightarrow B$ and $g: A \rightarrow B$ has the properties:

  \begin{enumerate}
  \item $e;\ f = e;\ g$
  \item For every arrow $e^\prime: X^\prime \rightarrow A$ satisfying $1$, there exists a unique arrow $k: X^\prime \rightarrow X$ such that $k;\ e = e^\prime$.
  \end{enumerate}

  Universal constructions describe a class of objects and arrows sharing a common property and pick out the terminal objects when this class is considered as a category.
  For equalizers, the common property is that $e'; f = e'; g$ for a class of arrows, and the terminal objects are the equalizer arrows; that is, the $e$ with a unique arrow from every other $e'$.

\item[]
\item[1.7.4.1]
  

\item[]
\item[1.7.4.2]
  Every equalizer $e : B \rightarrow C$ of arrows from $C$ to $D$ is monic because if we have arrows $k : A \rightarrow B$ and $k' : A \rightarrow B$ such that $e \circ k = e \circ k'$ then we are guaranteed that $k = k'$ because there is exactly one unique arrow $k$.
  (To compare to Pierce's definition, the arrow $e'$ in the equalizer diagram would be relabeled $e \circ k$.)
  
\item[]
\item[1.7.4.3]
  
\end{enumerate}
\end{document}
