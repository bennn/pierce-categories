\documentclass{article}
%% Chapter 2 Section 4: Adjoints

\usepackage{amsmath}
\usepackage{palatino}
\usepackage{tikz}

\newcommand{\one}{\mathbf{1}}
\newcommand{\cat}{\mathbf{C}}

\begin{document}

\begin{enumerate}
\item [2.4.5]
  \begin{itemize}
  \item 
    Not certain about this, but I think the unit diagram is:
    \begin{center}
      \begin{tikzpicture}
        \node (1) {$\one$};
        \node [right of=1, xshift=1cm] (2) {$\one$};
        \node [below of=2, yshift=-1cm] (3) {$\one$};
        
        \draw (1) -- node [above] {$\iota$} (2);
        \draw (1) -- node[left] {$f$} (3);
        \draw (2) -- node [right] {$T~f^{\#}$} (3);
      \end{tikzpicture}
    \end{center}
    The natural transformation $\iota : \one \rightarrow \one$ is defined by Pierce thus constraining the top two nodes and the functor $T : \cat \rightarrow \one$ is the right adjoint, constraining the bottom node to be $\one$.
    The category $\cat$ is only involved with the function $f^{\#}$.
    It acts the single $\cat$-object obtained by lifting the unique object of $\one$ into $\cat$.

    Not sure how to identify the initial object or the existence of unique arrows.
  \item 
  \end{itemize}
\item[]
\item[2.4.7]
\end{enumerate}
\end{document}
